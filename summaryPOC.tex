This project proposes the construction of a demonstrator for a new type of Positron Emission TOF Apparatus using Liquid xenOn (PETALO). The detector is based in the  Liquid Xenon Scintillating Cell (LXSC). The cell is a box filled with liquid xenon (LXe) whose transverse dimensions are chosen to optimize packing and with a thickness optimized to contain a large fraction of the incoming photons. The entry and exit faces of the box (relative to the incoming gammas direction) are instrumented with large silicon photomultipliers (SiPMs), coated with a wavelength shifter, tetraphenyl butadiene (TPB). The non-instrumented faces are covered by reflecting Teflon coated with TPB. Monte Carlo Studies show that the LXSC can display an energy resolution of 5\% FWHM, much better than that of conventional solid scintillators such as LSO/LYSO. The LXSC can measure the interaction point of the incoming photon with a resolution in the three coordinates of 1 mm. The very fast scintillation time of LXe (2 ns) and the availability of suitable sensors and electronics permits a coincidence resolution time (CRT) in the range of 100-200 ps, again much better than any current PET-TOF system. The proposed demonstrator consists in 24 LXSC of 30 x 30 x 30 mm3, deployed in a ring of  230 mm diameter. The ring is immersed into a cryostat. The entry and exit faces of each LXSC are instrumented with a matrix of 4x4 SiPMs of 6mm2. Therefore each LXSC has 32 channels. Two LXSC are read out by a 64-channel TOFPET ASIC integrating signal amplification and discrimination circuitry and high-performance TDCs for each channel, featuring 25 ps r.m.s intrinsic resolution and fully digital output. The purpose of this demonstrator is to show that the PETALO concept may offer a break-through in PET technology, thanks to the extraordinary combination of excellent energy and CRT resolution and low cost. 